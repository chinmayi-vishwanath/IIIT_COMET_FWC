\documentclass[12pt]{article}

\usepackage{amsmath,amssymb}
\usepackage{geometry}
\usepackage{graphicx}

\geometry{margin=1in}

\begin{document}

% ================= HEADER =================
\begin{minipage}[t]{0.35\textwidth}
    \vspace{0pt}
    \includegraphics[width=0.85\linewidth]{logo.png}
\end{minipage}
\hfill
\begin{minipage}[t]{0.6\textwidth}
    \vspace{9pt}
    \raggedleft
    \textbf{Name: Chinmayi V V}\\
    \textbf{ID: COMETFWC041}\\
    \textbf{Date: 03 Feb 2026}
\end{minipage}

\vspace{1.2cm}

\begin{center}
\textbf{\Large International Mathematical Olympiad Problems (1971--1973)}
\end{center}

\vspace{0.8cm}

% ================= 1971 =================
\section*{Thirteenth International Olympiad, 1971}

\textbf{1971/1.}
Prove that the assertion is true for $n=3$ and $n=5$, and false for every other natural number $n>2$.
If $a_1,a_2,\dots,a_n$ are real numbers, then
$(a_1-a_2)(a_1-a_3)\cdots(a_1-a_n)
+(a_2-a_1)(a_2-a_3)\cdots(a_2-a_n)
+\cdots
+(a_n-a_1)(a_n-a_2)\cdots(a_n-a_{n-1}) \ge 0$.

\medskip
\textbf{1971/2.}
Let $P_1$ be a convex polyhedron with vertices $A_1,A_2,\dots,A_9$.
For $i=2,3,\dots,9$, let $P_i$ be obtained by translating $P_1$ so that $A_1$ moves to $A_i$.
Prove that at least two of the polyhedra have a common interior point.

\medskip
\textbf{1971/3.}
Prove that the set of integers of the form $2^k-3$ for $k\ge2$
contains an infinite subset whose members are pairwise relatively prime.

\medskip
\textbf{1971/4.}
All faces of tetrahedron $ABCD$ are acute.
Let $X,Y,Z,T$ lie on edges $AB,BC,CD,DA$ respectively.
If $\angle DAB + \angle BCD \ne \angle CDA + \angle ABC$, show that no shortest polygonal path exists.
If equality holds, show that infinitely many shortest paths exist.

\medskip
\textbf{1971/5.}
For every natural number $m$, prove that there exists a finite set $S$ in the plane such that for each $A$ in $S$, exactly $m$ points of $S$ are at unit distance from $A$.

\medskip
\textbf{1971/6.}
Let $A=(a_{ij})$ be an $n\times n$ matrix of non-negative integers.
If $a_{ij}=0$ implies the sum of the $i$th row and $j$th column is at least $n$,
prove that the sum of all entries is at least $n^2/2$.

% ================= 1972 =================
\section*{Fourteenth International Olympiad, 1972}

\textbf{1972/1.}
From any set of ten distinct two-digit numbers, prove that two disjoint subsets can be chosen with equal sum.

\medskip
\textbf{1972/2.}
Prove that if $n\ge4$, every cyclic quadrilateral can be dissected into $n$ cyclic quadrilaterals.

\medskip
\textbf{1972/3.}
Let $m,n$ be non-negative integers.
Prove that $(2m)!(2n)!/(m!n!(m+n)!)$ is an integer.

\medskip
\textbf{1972/4.}
Find all positive real solutions $(x_1,\dots,x_5)$ satisfying
$(x_1^2-x_3x_5)(x_2^2-x_3x_5)\le0$,
$(x_2^2-x_4x_1)(x_3^2-x_4x_1)\le0$,
$(x_3^2-x_5x_2)(x_4^2-x_5x_2)\le0$,
$(x_4^2-x_1x_3)(x_5^2-x_1x_3)\le0$,
$(x_5^2-x_2x_4)(x_1^2-x_2x_4)\le0$.

\medskip
\textbf{1972/5.}
Let $f$ and $g$ satisfy $f(x+y)+f(x-y)=2f(x)g(y)$.
If $f$ is not identically zero and $|f(x)|\le1$, prove that $|g(y)|\le1$.

\medskip
\textbf{1972/6.}
Given four distinct parallel planes, prove that a regular tetrahedron exists with one vertex on each plane.

% ================= 1973 =================
\section*{Fifteenth International Olympiad, 1973}

\textbf{1973/1.}
Let $O$ be the origin and $P_1,\dots,P_n$ lie in a plane on one side of $O$.
Prove that $|OP_1|+\cdots+|OP_n|\ge|OP_1+\cdots+OP_n|$.

\medskip
\textbf{1973/2.}
Determine whether a finite set $M$ exists such that for any $A,B$ in $M$,
there exist $C,D$ in $M$ with $AB\parallel CD$ and $AB$ not coincident with $CD$.

\medskip
\textbf{1973/3.}
Let $a,b$ be real numbers such that $x^4+ax^3+bx^2+ax+1=0$ has a real solution.
Find the minimum of $a^2+b^2$.

\medskip
\textbf{1973/4.}
A soldier must scan an equilateral triangular region with a detector whose range equals half the altitude.
Starting from one vertex, find the shortest path covering the region.

\medskip
\textbf{1973/5.}
Let $G$ be the set of non-constant linear functions $f(x)=ax+b$ satisfying:
closure under composition, closure under inverse, and each function has a fixed point.
Prove that there exists a real number $k$ such that $f(k)=k$ for all $f$ in $G$.

\medskip
\textbf{1973/6.}
Let $0<a_i<1$ and $b_1+\cdots+b_n\ge\ln(a_1+\cdots+a_n)$.
Find all real numbers $b_1,\dots,b_n$ satisfying this.

\end{document}
